\chapter*{CONCLUSIONES Y RECOMENDACIONES}
\addcontentsline{toc}{chapter}{CONCLUSIONES}
\section{Comentarios generales}

\begin{indentar}
Luego de la elaboraci\'on del PSM de cada m\'odulo con su correspondiente PIM, se llev\'o a cabo la transformaci\'on deseada del c\'odigo, s\'olo del dise\~no del dominio.

En este punto es necesario comentar acerca de las herramientas que usamos para la generaci\'on deseada del PSM al c\'odigo, en primera instancia deseamos utilizar una herramienta llamada Magic Draw\footnote{\url{http://www.magicdraw.com/}}, una herramienta que no s\'olo nos gener\'o las clases propias del dominio, si no tambi\'en el correspondiente DDL\footnote{Data Definition Language} de la Base de Datos que nosotros hab\'iamos seleccionado; sin embargo, la curiosidad intelectual nos llev\'o a utilizar otra herramienta CASE Open Source para utilizar las metodolog\'ias MDA y MERODE en conjunto, y la que se seleccion\'o fue StarUML debido a sus singulares caracter\'isticas de las que en su momento deseamos aprovecharnos, pero existi\'o la gran desventaja que esta herramienta solo nos proporcion\'o la transformaci\'on del PSM al c\'odigo java.

La \'unica ventaja que se pudo sacar de esta herramienta fue la de conocer el hecho que se pueden implementar cartuchos para la transformaci\'on del PSM al c\'odigo, no s\'olo de la parte del dominio del sistema, si no tambi\'en de la capa controlador, que en el caso nuestro, fue implementado seg\'un la arquitectura J2EE.

Las ''transformaciones manuales'' que se realizaron en la mayor\'ia del proyecto, nos ayud\'o a la estandarizaci\'on del c\'odigo de la misma, conservando una gran similitud en la parte del c\'odigo en el lado del servidor y del cliente, conservando el patr\'on de dise\~no MVC 2 mencionado como requerimiento no funcional.
\end{indentar}
